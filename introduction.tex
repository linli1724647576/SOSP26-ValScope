\section{Introduction}
\label{sec:intro}

% DBMS Logical bug 以及 Detect Logical Bug 的方法
Database Management Systems (DBMSs) play a critical role in applications such as online banking and e-commerce~\cite{florescu1998database, zhou2020database}. 
Like other large-scale systems, DBMSs involve complex code logic and a wide range of functionalities, making them prone to bugs during both development and maintenance~\cite{adya1999weak, jiang2023dynsql, jung2019apollo}. 
Among these, logical bugs are particularly critical as they silently lead to incorrect query results in DBMSs~\cite{rigger2020detecting, kamm2023testing, rigger2020detecting}.
To detect logical bugs, existing approaches~\cite{rigger2020detecting,rigger2020finding,rigger2020testing,song2023testing,tang2023detecting,jiang2024detecting,song2024detecting,hao2023pinolo,deng2025detecting} generate SQL queries to test DBMSs and check whether the produced results follow the expectations.
One of the fundamental technical challenges in this process is to define what constitutes a correct result for a given query, which is a classical problem in software testing—known as the test oracle problem~\cite{howden2006theoretical}. 
To address the test oracle problem, many methods have been proposed, such as differential testing~\cite{bati2007genetic,slutz1998massive} and oracle-guided synthesis~\cite{rigger2020testing}. 
However, the most mainstream and effective approach currently is metamorphic testing (MT), which establishes expected semantic relations between pairs of systematically transformed queries~\cite{hao2023pinolo,lin2025qtran}. 

% 现有metamorphic testing的局限性
Although MT has become the most effective paradigm for detecting logical bugs in DBMSs, existing MT approaches~\cite{rigger2020detecting,rigger2020finding,rigger2020testing,song2023testing,tang2023detecting,jiang2024detecting,song2024detecting,hao2023pinolo,deng2025detecting} still struggle to capture many subtle semantic inconsistencies.  
Current MT approaches can be broadly divided into two categories: those based on \textbf{equivalent metamorphic relations} and those based on \textbf{approximation metamorphic relations}.
The first category, represented by \textsc{NoREC}~\cite{rigger2020detecting} and \textsc{TLP}~\cite{rigger2020finding}, defines the MR as strict output equivalence between the original and transformed queries.
\textsc{NoREC} transforms an optimizable query into a non-optimizable form to verify that both versions produce identical results, while \textsc{TLP} partitions query predicates into multiple subqueries whose combined outputs are expected to equal the original result.  
While these equivalence-based frameworks have proven effective in detecting optimizer and predicate-related inconsistencies, they remain insufficient to expose many deeply hidden bugs~\cite{hao2023pinolo}.  
This limitation arises because, under a constrained mutation space, the pair of equivalent queries often still share the same buggy operators or functions, ultimately producing identical—but incorrect—results.  
Consequently, such approaches are restricted to surface-level correctness and fail to uncover deeper semantic flaws that preserve structural equivalence yet violate relational semantics internally.

The second category, exemplified by \textsc{Pinolo}~\cite{hao2023pinolo}, generalizes this paradigm through set-semantic approximation.  
Instead of requiring exact equality, \textsc{Pinolo} relaxes query predicates to generate over- and under-approximate variants and checks whether the resulting sets satisfy inclusion or containment relations.  
This relaxation significantly expands the detectable bug space, allowing the detection of subtle semantic inconsistencies that equivalence-based approaches often overlook.  
However, approximation at the set level still neglects finer-grained deviations that occur within the same tuple set—such as erroneous aggregation, numeric computation, or ordering—where the result set remains unchanged but the value semantics are corrupted.  

\lin{add a motivating example.}
% 举个逻辑错误的例子作为motivating example,为什么其他工作detect不到这个bug

% Key insight
In many cases, logical bugs do not alter the result set but silently corrupt the underlying value semantics of the query output.  
To capture such hidden inconsistencies more effectively, we introduce the notion of \textbf{value-semantic approximation}, which explicitly models the direction and magnitude of value changes through monotonic relationships.  
By seamlessly integrating set-semantic approximation with value-semantic approximation, we establish \textbf{a unified and comprehensive SQL query approximation model} for logical bug detection—enabling the discovery of more nuanced and deeply hidden semantic faults that existing MT frameworks cannot capture.
\lin{explain why the bug can be detected by the approximation model.}

% Implement - Our approach
Based on the proposed SQL query approximation model, we design a novel approach, \toolname\, which detects logical bugs in DBMSs by jointly reasoning about both set-semantic and value-semantic approximation.  
\toolname\ follows a generate–mutate–verify paradigm.  
It first generates syntactically valid and semantically diverse SQL queries and then systematically mutates them according to the defined approximation mutators.  
For each pair of original and mutated queries, \toolname\ performs approximation propagation analysis to determine how local semantic mutations (e.g., changing \texttt{MAX} to \texttt{MIN} or relaxing predicates) affect global query behavior along the SQL abstract syntax tree (AST).
Finally, both queries are executed on the target DBMS, and their outputs are compared against the predicted approximation relation.  
Any violation of the expected set- or value-level relationship indicates a potential logical bug.  
By integrating value-semantic reasoning with structural approximation, \toolname\ uncovers a wider range of subtle inconsistencies that remain invisible to prior MT frameworks.

% Evaluation
We implemented our approach as a practical DBMS testing tool and evaluate it on \textcolor{red}{6} widely-used and extensively-tested DBMSs, MySQL xxx.
In total , we find xxx logical bugs
% 参考 EET 或者 Pinolo 的写法。

% Contribution
Overall, we make the following contributions:
\begin{itemize}
	\item We propose a unified model, SQL query approximation, that combines set-semantic and value-semantic reasoning to detect logical bugs in DBMSs.  
	
	\item We develop and implement a novel metamorphic testing framework, \toolname, which systematically generates, mutates, and verifies SQL queries based on the proposed approximation relations, effectively identifying logical bugs in DBMSs.  
	
	\item We evaluate \toolname on \textcolor{red}{6} real-world DBMSs. In total, we found xxx logical bugs. To further facilitate research on DBMS testing, we open-source the tool at \lin{add a url}.  
\end{itemize}




