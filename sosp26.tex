%%
%% This is file `sample-sigconf-biblatex.tex',
%% generated with the docstrip utility.
%%
%% The original source files were:
%%
%% samples.dtx  (with options: `sigconf-biblatex')
%% 
%% IMPORTANT NOTICE:
%% 
%% For the copyright see the source file.
%% 
%% Any modified versions of this file must be 
%% with new filenames distinct from sample-sigconf-biblatex.tex.
%% 
%% For distribution of the original source see the terms
%% for copying and modification in the file samples.dtx.
%% 
%% This generated file may be distributed as long as the
%% original source files, as listed above, are part of the
%% same distribution. (The sources need not necessarily be
%% in the same archive or directory.)
%%
%%
%% Commands for TeXCount
%TC:macro \cite [option:text,text]
%TC:macro \citep [option:text,text]
%TC:macro \citet [option:text,text]
%TC:envir table 0 1
%TC:envir table* 0 1
%TC:envir tabular [ignore] word
%TC:envir displaymath 0 word
%TC:envir math 0 word
%TC:envir comment 0 0
%%
%%
%% The first command in your LaTeX source must be the \documentclass
%% command.
%%
%% For submission and review of your manuscript please change the
%% command to \documentclass[manuscript, screen, review]{acmart}.
%%
%% When submitting camera ready or to TAPS, please change the command
%% to \documentclass[sigconf]{acmart} or whichever template is required
%% for your publication.
%%
%%
%\documentclass[sigconf,review,natbib=false]{acmart}
%\documentclass[sigconf,natbib=false,screen]{acmart}
%\documentclass[sigconf,screen]{acmart}
%\documentclass[acmsmall,screen,review,anonymous]{acmart}
\documentclass[sigplan,10pt]{acmart}
\renewcommand\footnotetextcopyrightpermission[1]{}


\usepackage{graphicx}
\usepackage{algorithm}
\usepackage{algpseudocode}
\usepackage{booktabs} 
\usepackage{threeparttable}
\usepackage{multirow}
\usepackage{enumitem}
\usepackage{colortbl}
\usepackage[most]{tcolorbox}
\usepackage{caption}
\usepackage{balance}
\usepackage[normalem]{ulem}
\usepackage{xcolor}
\pagestyle{empty}

\usepackage{amsthm}

% 定义 Definition 环境
\theoremstyle{definition}
\newtheorem{definition}{Definition}[section]
\newtheorem{example}{Example}[section]

% ----------- 改进样式 -----------
\algrenewcommand\algorithmicprocedure{\textbf{Procedure}} % 大写开头
\algrenewcommand\algorithmicif{\textbf{if}}
\algrenewcommand\algorithmicthen{}
\algrenewcommand\algorithmicelse{\textbf{else}}
\algrenewcommand\algorithmicfor{\textbf{for}}
\algrenewcommand\algorithmicforall{\textbf{for all}}
\algrenewcommand\algorithmicdo{\textbf{do}}
\algrenewcommand\algorithmicend{\textbf{end}}
\algrenewcommand\algorithmicrequire{\textbf{Input:}}
\algrenewcommand\algorithmicensure{\textbf{Output:}}
\renewcommand{\thealgorithm}{\arabic{algorithm}}
\floatname{algorithm}{Algorithm}
\algdef{SE}[PROCEDURE]{Procedure}{EndProcedure}[2]{\textbf{procedure} #1(#2)}{\textbf{end procedure}}
% 调整行距和缩进
\makeatletter
\renewcommand{\ALG@beginalgorithmic}{\small}
\makeatother


\usepackage{pifont}
\usepackage{booktabs}
\usepackage{cleveref}
\crefname{section}{§}{§§}
\Crefname{section}{§}{§§}

\newcommand{\mytodoblue}[1]{\textcolor{blue}{\ding{46}~{\sf}~#1}}
\newcommand{\mytodored}[1]{\textcolor{red}{\ding{46}~{\sf}~#1}}
\newcommand{\mytodogreen}[1]{\textcolor{mygreen}{\ding{46}~{\sf}~#1}}
\newcommand{\mytodoorange}[1]{\textcolor{orange}{\ding{46}~{\sf}~#1}}
\newcommand{\mytodocyan}[1]{\textcolor{cyan}{\ding{46}~{\sf}~#1}}
\newcommand{\mytodopink}[1]{\textcolor{purple}{\ding{46}~{\sf}~#1}}
\usepackage{pgffor}
\usepackage{pifont} % For checkmark and cross symbols

\usepackage{xcolor,pifont}

% 自定义彩色打勾和打叉命令
\newcommand{\cmark}{\textcolor{green!60!black}{\ding{51}}}   % 绿色打勾
\newcommand{\xmark}{\textcolor{red}{\ding{55}}}   % 红色打叉


% to show s or not
\newif\ifshowcomments
%\showcommentsfalse
\showcommentstrue
\ifshowcomments
\newcommand{\wu}[1]{\mytodored{[wu: #1]}}
\newcommand{\cp}[1]{\mytodored{[cp: #1]}}
%\newcommand{\zhang}[1]{\mytodoorange{[zhang: #1]}}
\newcommand{\lin}[1]{\mytodored{[lin: #1]}}
%\newcommand{\other}[1]{\mytodoblue{[other: #1]}}
\else
\newcommand{\wu}[1]{}
\newcommand{\cp}[1]{}
%\newcommand{\zhang}[1]{}
\newcommand{\lin}[1]{}
%\newcommand{\other}[1]{}
\fi

\newif\ifshowrevisioncomments
\showcommentsfalse
%\showrevisioncommentstrue
\showrevisioncommentsfalse
\ifshowrevisioncomments
\newcommand{\revision}[1]{\textcolor{blue}{{#1}}}
\newcommand{\deletion}[1]{\textcolor{red}{{\sout{#1}}}}
\else
\newcommand{\revision}[1]{#1}
\newcommand{\deletion}[1]{}
\fi


\newcommand{\toolname}{\textsc{ValScope}}
\newcommand{\reported}{24}
\newcommand{\confirmed}{16}

\bibliographystyle{ACM-Reference-Format}

\settopmatter{printacmref=false}
\setcopyright{none}
\renewcommand\footnotetextcopyrightpermission[1]{}
\pagestyle{plain}


\begin{document}
\newtcolorbox{mybox}[2][]{
  colframe = black,
  colback  = gray!20,
  coltitle = black,
  left = 1mm,
  right = 1mm,
  top = 1mm,
  bottom = 1mm,
  boxsep=5pt,
  arc=2mm,
  title=#2,
  #1
}

\title{ValScope: Value-Semantics-Aware Metamorphic Testing for Detecting Logical Bugs in DBMSs}


\begin{abstract}
    Database Management Systems (DBMSs) are crucial for data processing in many large-scale applications. 
    Detecting logical bugs in DBMSs is a challenging task, as traditional testing methods struggle to define what constitutes a "correct" query result. 
    Metamorphic testing (MT) has emerged as a prominent solution to this problem, by establishing expected metamorphic relations between pairs of systematically transformed queries. 
    However, existing MT approaches, which rely on \textit{equivalent} or \textit{set-semantic} relations, are limited in their ability to detect subtle semantic inconsistencies that do not alter the result set but corrupt the value semantics, such as incorrect aggregation, sorting, or numeric computation.
	
	In this paper, we introduce a unified \textit{SQL query approximation} model that combines both set-semantic and value-semantic reasoning to capture more nuanced logical bugs. 
	The core idea is to not only reason about the inclusion or equivalence of result sets but also to model how changes in value semantics—such as the direction and magnitude of numeric changes—affect query correctness. 
	Based on this model, we propose our approach, \toolname, which generates and mutates SQL queries according to predefined mutators, performing approximation propagation analysis to track how local semantic changes propagate and influence global query behavior.
	We evaluate \toolname on \textcolor{red}{6} widely-used DBMSs and identify a total of \textcolor{red}{xxx} logical bugs, many of which were previously undetected by existing approaches. 
	Our results demonstrate that \toolname\ significantly expands the scope of detectable logical bugs, capturing a broader range of bugs that remain invisible to current MT approaches.
\end{abstract}

\settopmatter{printfolios=true}
\maketitle
\pagestyle{plain}
\section{Introduction}
\label{sec:intro}


% 举个逻辑错误的例子作为motivating example

%\section{Background}
\label{sec:Background}

\subsection{Database Management Systems and SQL}

\textbf{Database Management Systems.} 
Database Management Systems (DBMSs) are fundamental to modern software ecosystems, offering systematic mechanisms for storing, organizing, and accessing large volumes of structured data. 
This work focuses on \textit{relational DBMSs}, which manage data according to the relational model~\cite{codd1970relational}. 
In such systems, data is organized into tables, each containing tuples (records) that represent real-world entities. 
Mathematically, each table corresponds to a relation defined over a finite set of attributes, and the database comprises a collection of such relations. 
In practice, DBMSs allow developers to efficiently insert, update, delete, and query data through structured interfaces, ensuring both data consistency and accessibility.

\textbf{SQL.} 
Structured Query Language (SQL)~\cite{chamberlin1974sequel} is the standard interface for interacting with relational DBMSs. 
It provides a unified syntax for defining, manipulating, and querying data, and can be broadly classified into four categories: 
Data Definition Language (DDL), Data Manipulation Language (DML), Data Query Language (DQL), and Data Control Language (DCL). 
Among them, DQL, represented primarily by the \texttt{SELECT} statement, forms the core of data retrieval in relational systems. 
DQL is used to specify the desired information without explicitly describing how it should be obtained, reflecting SQL’s declarative nature. 
The \texttt{SELECT} statement supports rich semantics including filtering, grouping, aggregation, and joining across multiple relations, making it the most fundamental yet semantically complex component of SQL. 
In this work, we focus on detecting logical bugs in DQL, as they constitute the majority of real-world query workloads and are crucial to ensuring the correctness and reliability of DBMS query processing.

\subsection{Logical Bugs in DBMSs}

\textbf{Logical Bugs.} 
Logical bugs are one of the most critical types of bugs in DBMSs, silently causing incorrect query results without triggering system crashes~\cite{hao2023pinolo, lin2025qtran}. 
Unlike crash bugs that exhibit obvious failures, logical bugs corrupt query outputs in subtle ways, posing serious risks to data integrity and application reliability. 
%Detecting such bugs is particularly challenging due to the test oracle problem—it is often impossible to know the correct result of a complex SQL query in advance~\cite{hao2023pinolo}. 
%Hence, an effective DBMS testing framework must infer correctness relations between queries without relying on manually crafted oracles.

\textbf{Existing Detection Approaches.} 
Existing research on detecting logical bugs mainly relies on automated testing techniques. 
However, designing an effective testing framework is challenging due to the \textbf{test oracle problem}—it is difficult to determine the correct result of a complex SQL query for comparison. 
To address this issue, researchers have explored three main categories of approaches~\cite{hao2023pinolo}. 
The first is differential testing~\cite{bati2007genetic,slutz1998massive}, which executes the same query across multiple DBMSs and compares their outputs; inconsistencies reveal potential logic flaws, but dialect differences and heterogeneous semantics often limit its applicability.  
The second, oracle-guided synthesis~\cite{rigger2020testing}, generates queries expected to return a specific pivot row and reports an error when the row is missing, but it only captures localized issues and fails to expose deeper semantic inconsistencies.  
The third and most prominent category is metamorphic testing, which establishes expected semantic relations between pairs of systematically transformed queries.  
This approach enables \textit{oracle-free} verification, offering better scalability and generality.  
We will provide an in-depth analysis in the following section.


\subsection{Metamorphic Testing}
In recent years, metamorphic testing (MT) has become the most effective and widely adopted approach for detecting logical bugs in DBMSs~\cite{hao2023pinolo,lin2025qtran}.  
The core idea of MT is to construct multiple SQL statements whose results are expected to satisfy a specific relation, known as a metamorphic relation (MR).  
When the actual query results violate this relation, it indicates that at least one query triggers a logical bug in the tested DBMS.  

\textbf{Equivalent MR.}
Traditional MT, such as \textsc{NoREC}~\cite{rigger2020detecting} and \textsc{TLP}~\cite{rigger2020finding}, define the metamorphic relation as strict equivalence between the outputs of the original and transformed queries.  
For example, \textsc{NoREC} transforms an optimizable query into a non-optimizable one, while \textsc{TLP} decomposes a predicate into multiple subqueries and merges their results to ensure equality. 
Although this equivalence-based strategy effectively bypasses the lack of ground truth, it constrains the search space of test cases: the transformations typically preserve all operators, functions, and predicates of the original query, merely changing structural forms.  
As a result, such approaches are limited to verifying surface-level correctness and often fail to reveal deeper semantic errors—particularly those that do not produce directly inconsistent result sets but still violate relational semantics internally~\cite{hao2023pinolo}.  

\textbf{Approximation MR.}
To overcome the over-restrictive nature of equivalence-based testing, \textsc{Pinolo}~\cite{hao2023pinolo} introduces the concept of set-semantic approximation relations, relaxing the equivalence assumption.
By expanding or constraining query predicates, it generates over- and under-approximate queries and judges correctness through inclusion or containment relations between result sets. 
This relaxation enables the detection of a broader spectrum of logical bugs, uncovering deeper semantic inconsistencies that equivalence-based approaches often miss~\cite{hao2023pinolo}.  

However, approximation at the set level still overlooks subtle semantic shifts that occur at the value level, such as incorrect aggregations, ordering, or numeric deviations that preserve the same tuple set but alter the meaning of the result.
Building on this insight, our work extends MT from set semantics to value semantics.  
We propose a new class of value-semantic approximation relations that reason about the direction and magnitude of result changes while maintaining structural consistency.  
By integrating set-semantic approximation with value-semantic approximation, our framework establishes a unified, multi-dimensional criterion for logical bug detection—enabling the discovery of more nuanced and deeply hidden semantic faults that existing MT frameworks cannot capture.


 
%\section{Approach}
%\label{sec:approach}

\section{SQL Query Approximation}

A fundamental challenge in DBMS testing lies in the test oracle problem:  
given two semantically related SQL queries, it is often unclear whether their outputs should be identical or follow a predictable metamorphic relationship.  
Traditional testing frameworks typically rely on strict equivalence or simple set inclusion checks, which are insufficient to capture subtle semantic deviations introduced by optimizer transformations or operator mutations~\cite{hao2023pinolo}.  
To address this limitation, we introduce a more comprehensive notion of SQL Query Approximation,  which unifies two complementary perspectives of query behavior: the \textbf{set-semantic} and \textbf{value-semantic} dimensions.  
The set-semantic dimension characterizes differences in the returned tuple sets, while the value-semantic dimension captures monotonic variations in the computed or aggregated values over those tuples.  
Together, these two dimensions form a unified framework for expressing and reasoning about semantic consistency, enabling the construction of more expressive test oracles that can detect both structural and value-level inconsistencies in DBMS behavior.

\subsection{Set-Semantic Approximation}

In this section, we first formalize the notion of approximation at the set level.  
This relation captures inclusion or containment among query result sets.

\begin{definition}[Set-Semantic Approximation Relation]
	\label{def:set_approximation}
	
	Given a database \( D \), let \( q_1 \) and \( q_2 \) be two SQL queries whose result sets are \( R(q_1, D) \) and \( R(q_2, D) \), respectively.  
	We say that \( q_1 \) is the \textit{set-based under-approximation} of \( q_2 \) over \( D \), denoted by  
	\( q_1 \preceq_D^s q_2 \), if and only if:
	\[
	R(q_1, D) \subseteq R(q_2, D)
	\]
	Conversely, \( q_1 \) is the \textit{set-based over-approximation} of \( q_2 \) over \( D \), denoted by  
	\( q_1 \succeq_D^s q_2 \), if and only if:
	\[
	R(q_1, D) \supseteq R(q_2, D)
	\]
	Here, \( R(q, D) \) represents the multi-set returned by evaluating query \( q \) on database \( D \),  
	and \( \subseteq \) and \( \supseteq \) denote inclusion and containment relations between two multi-sets.
\end{definition}


Intuitively, the set-semantic approximation forms a partial order over queries:  
\( q_1 \preceq_D^s q_2 \) means that \( q_1 \) produces a narrower or more restrictive result than \( q_2 \),  
while \( q_1 \succeq_D^s q_2 \) means that \( q_1 \) yields a broader or less restrictive result.  
These two relations are inverses of each other and together define the lattice of set-based approximations.


\begin{example}
	Consider a database \( D = \{t_1\} \), where \( t_1(c_1) = \{-1, 0, 1\} \).  
	Let the following queries be defined:
	\[
	q_1 : \text{\textbf{SELECT} } c_1 \text{ \textbf{FROM} } t_1 \text{ \textbf{WHERE} } c_1 \le 0
	\]
	\[
	q_2 : \text{\textbf{SELECT} } c_1 \text{ \textbf{FROM} } t_1 \text{\textbf{ WHERE} \textbf{TRUE}}
	\]
	\[
	q_3 : \text{\textbf{SELECT} } c_1 \text{ \textbf{FROM}} t_1 \text{ \textbf{WHERE} } c_1 < 0
	\]
	We have:
	\[
	R(q_3, D) = \{-1\}, \quad R(q_1, D) = \{-1, 0\}, \quad R(q_2, D) = \{-1, 0, 1\}
	\]
	Hence,
	\[
	R(q_3, D) \subseteq R(q_1, D) \subseteq R(q_2, D)
	\]
	which gives the approximation chain:
	\[
	q_3 \preceq_D^s q_1 \preceq_D^s q_2
	\]
	Intuitively, \( q_3 \) is a stricter version of \( q_1 \), and \( q_1 \) a stricter version of \( q_2 \),
	each progressively expanding the selection condition and thus broadening the result set.
\end{example}

\subsection{Value-Semantic Approximation}

To overcome the limitation of purely set-based inclusion, we extend the approximation relation from the result-set level to the value level. 
Unlike the set-based relation that focuses on tuple inclusion, the value-semantic relation captures the monotonic variation of target columns — the columns whose values are directly affected by functional or aggregation operations.  
This allows the framework to detect logical bugs where queries return identical tuples but diverge in their value semantics, such as incorrect computations in aggregation or updates.

\begin{definition}[Value-Semantic Approximation Relation]
	\label{def:value_approximation}
	Given a database \( D \), let \( q_1 \) and \( q_2 \) be two SQL queries whose result sets are 
	\( R(q_1, D) \) and \( R(q_2, D) \), respectively.  
	Let \( C_t \subseteq \text{Cols}(R(q_1, D)) \cap \text{Cols}(R(q_2, D)) \) denote the target columns
	whose values will be compared.  
	Let \( G \) denote the grouping or ordering basis, determined as follows:
	
	If the query contains a \texttt{GROUP BY} clause, \( G \) corresponds to the group-by keys.
	Otherwise, \( G \) represents a deterministic ordering over non-target columns 
	(e.g., primary key or lexicographic ordering of attributes) to align tuples for comparison.
	
	We say that \( q_1 \) is the \textit{value-based over-approximation} of \( q_2 \) over \( D \), denoted by 
	\( q_1 \succeq_D^v q_2 \), if and only if:
	\[
	\forall g \in G^*, \forall c \in C_t, \;
	V_{q_1}(g, c) \ge V_{q_2}(g, c)
	\]
	where \( G^* \) is the set of all comparable tuple groups under \( G \),  
	and \( V_q(g, c) \) denotes the value of column \( c \) in group \( g \) (or tuple position) 
	produced by query \( q \).
	
	Conversely, \( q_1 \) is the \textit{value-based under-approximation} of \( q_2 \), denoted by 
	\( q_1 \preceq_D^v q_2 \), if and only if:
	\[
	\forall g \in G^*, \forall c \in C_t, \;
	V_{q_1}(g, c) \le V_{q_2}(g, c)
	\]
	This definition unifies two cases: group-wise comparison for aggregation queries, 
	and order-aligned comparison for non-aggregated results.
\end{definition}


\noindent
Intuitively, the \textit{set-semantic approximation} (\( \preceq_D^s \)) describes inclusion of tuples,  
while the \textit{value-semantic approximation} (\( \preceq_D^v \)) 
reflects monotonicity among the values of corresponding tuples.  
\lin{Check why can coexist?}
In practice, the two forms often coexist:
\[
q_1 \preceq_D^s q_2 \;\wedge\; q_1 \preceq_D^v q_2
\]
which indicates that \( q_1 \) returns a subset of \( q_2 \)’s tuples 
and the corresponding values in the target columns are not larger.

\begin{example}
	Consider a table \( t_1(c_1, c_2) \) as follows:
	\[
	t_1 =
	\begin{array}{c|c}
		c_2 & c_1 \\
		\hline
		A & 10 \\
		A & 20 \\
		B & 5 \\
		B & 7
	\end{array}
	\]
	Let the following two queries be defined:
	\[
	q_1 : \text{\textbf{SELECT} } c_2, \text{ MAX}(c_1) 
	\text{ \textbf{FROM} } t_1 \text{ \textbf{GROUP BY} } c_2
	\]
	\[
	q_2 : \text{\textbf{SELECT} } c_2, \text{ MIN}(c_1) 
	\text{ \textbf{FROM} } t_1 \text{ \textbf{GROUP BY} } c_2
	\]
	The results are:
	\[
	R(q_1, D) =
	\begin{array}{c|c}
		c_2 & \text{MAX}(c_1) \\
		\hline
		A & 20 \\
		B & 7
	\end{array}
	\quad
	R(q_2, D) =
	\begin{array}{c|c}
		c_2 & \text{MIN}(c_1) \\
		\hline
		A & 10 \\
		B & 5
	\end{array}
	\]
	Under the grouping basis \( G = \{c_2\} \) and target column \( C_t = \{c_1\} \),
	we have for each \( g \in G^* = \{\text{A}, \text{B}\} \):
	\[
	V_{q_1}(g, c_1) \ge V_{q_2}(g, c_1)
	\]
	Hence, \( q_1 \succeq_D^v q_2 \).  
	Intuitively, both queries return identical group sets (thus \( q_1 \equiv_D^s q_2 \)), 
	but differ monotonically in their value semantics:  
	the aggregated value of \( q_1 \) in each group is no smaller than that of \( q_2 \).
\end{example}

\subsection{Approximation Propagation}

The approximation relations introduced in the previous section capture the semantic correspondence between two complete SQL queries by comparing their result sets or value outputs.  
However, in practical DBMS testing, a mutation usually affects only a local part of the query---for instance, 
a predicate, an operator, or an aggregation function---rather than the entire query.  
To understand how such a local change influences the final query result, we extend the discussion from the semantic level of full-query comparison to the structural level of SQL.  
Specifically, we define the concept of \textit{approximation propagation}, 
which describes how a local approximation relation established at one node of the query’s abstract syntax tree (AST)
can be transmitted through its parent operators and clauses, 
thereby determining how a single mutation impacts the overall approximation behavior of the query.


\begin{definition}[Approximation Propagation]
	\label{def:approximation_propagation}
	
	Let \( D \) be a database, and let \( n_1, n_2 \) denote two semantically comparable
	nodes (e.g., subqueries, predicates, or expressions) in the SQL AST.
	We use the unified notation \( n_1 \preceq_D^\alpha n_2 \)
	to represent an \textit{approximation relation} of type \( \alpha \in \{s, v\} \),
	where \( s \) and \( v \) correspond to the set-semantic and value-semantic levels, respectively.
	The relations defined in \S\ref{def:set_approximation} and \S\ref{def:value_approximation}
	describe query-level approximations between complete queries.
	In contrast, approximation propagation extends these relations to the structural level,
	capturing how local approximations between AST nodes can influence or induce
	approximations at higher layers of the query.
	
%	Formally, let \( \sigma(op) \in \{+1, -1\} \) denote the \textit{monotonic direction} of an operator \( op \):
%	\(+1\) means the operator preserves the approximation direction (monotone increasing),
%	while \(-1\) means it reverses the direction (monotone decreasing or negating).
%	Then, the propagation of \( n_1 \preceq_D^\alpha n_2 \) follows the four canonical forms:

	Formally, each operator \( op \) is characterized by two semantic properties: a \textit{mapping} \((\alpha_{in} \!\rightarrow\! \alpha_{out})\), which specifies how the operator transforms between set-level and value-level semantics, and a \textit{direction} \(\sigma(op)\!\in\!\{+1, -1\}\), where \(+1\) indicates that the operator preserves the approximation direction (monotone increasing) and \(-1\) indicates that it reverses the direction (monotone decreasing or negating). 
	Based on these properties, the propagation of \( n_1 \preceq_D^\alpha n_2 \) can be classified into four canonical forms:

	\begin{itemize}[leftmargin=10pt]
		
		\item \textbf{(Set $\rightarrow$ Set)}:
		If a subquery or predicate \( p_1 \preceq_D^s p_2 \) is embedded under a higher-level
		set operator \( op_s \) (e.g., \texttt{EXISTS}, \texttt{NOT EXISTS}, logical \texttt{NOT}),
		then the resulting relation satisfies:
		\[
		R(n_1, D) \preceq_D^{s \cdot \sigma(op_s)} R(n_2, D)
		\]
		where operators such as \texttt{EXISTS} are monotone increasing (\( \sigma=+1 \)),
		while \texttt{NOT EXISTS} or \texttt{NOT} are monotone decreasing (\( \sigma=-1 \)),
		reversing inclusion (\( \subseteq \leftrightarrow \supseteq \)).
		
		\item \textbf{(Set $\rightarrow$ Value)}:
		If an aggregation or mapping function \( f \) is applied to two relations
		that satisfy \( R(n_1, D) \preceq_D^s R(n_2, D) \),
		then the corresponding value-level results satisfy:
		\[
		f(R(n_1, D)) \preceq_D^{v \cdot \sigma(f)} f(R(n_2, D))
		\]
		where \( \sigma(f)=+1 \) for monotone-increasing functions
		(e.g., \texttt{MAX}, \texttt{SUM}, \texttt{COUNT}),
		and \( \sigma(f)=-1 \) for monotone-decreasing ones (e.g., \texttt{MIN}).
		
		\item \textbf{(Value $\rightarrow$ Value)}:
		If an expression or scalar operator \( op_v \) is transformed to another form
		with monotonic direction \( \sigma(op_v) \),
		then the resulting value-level outputs satisfy:
		\[
		V_{n_1}(g, c) \preceq_D^{v \cdot \sigma(op_v)} V_{n_2}(g, c)
		\]
		This covers arithmetic transformations (\texttt{+}, \texttt{*2} with \(c>0\)) and functional ones (\texttt{MAX}$\rightarrow$\texttt{MIN}).
		
		\item \textbf{(Value $\rightarrow$ Set)}:
		If a value expression \( V_{n} \) feeds into a predicate or filtering operator \( op_s \),
		and \( V_{n_1} \preceq_D^v V_{n_2} \),
		then the induced output relations satisfy:
		\[
		R(n_1, D) \preceq_D^{s \cdot \sigma(op_s)} R(n_2, D)
		\]
		where \( \sigma(op_s)=+1 \) for monotone-increasing predicates
		(e.g., \texttt{x > c}, where larger values of \texttt{x} make the condition more likely to hold and thus expand the result set),
		and \( \sigma(op_s)=-1 \) for monotone-decreasing ones
		(e.g., \texttt{x < c} or \texttt{NOT EXISTS}, where larger values of \texttt{x} make the condition less likely to hold, causing the result set to shrink).
		
		
	\end{itemize}
\end{definition}

\noindent
\textbf{Remark.}
In this definition, \( n_1 \) and \( n_2 \) are not restricted to complete queries.
They can represent corresponding subqueries, expressions, or predicates
within a single query or across two query variants.
The relation \( \preceq_D^\alpha \) thus captures how a local semantic approximation
propagates through SQL operators according to their monotonic behavior,
bridging the value- and set-level semantics within the same unified framework.

Intuitively, the propagation mechanism provides the semantic bridge between 
\textit{tuple-level inclusion} and \textit{value-level monotonicity}.
Set-based approximations can trigger value changes through monotone operators,
while value-based changes can, in turn, alter the query result set when the affected values participate in predicates.
This bidirectional propagation enables comprehensive reasoning over multi-layer SQL dependencies.


\begin{example}[Set $\rightarrow$ Value Propagation]
	Consider two queries over a table \( t_1(c_1, c_2) \):
	\[
	q_1 : \textbf{SELECT MAX}(c_1) \textbf{ FROM } t_1 \textbf{ WHERE } c_2 < 100
	\]
	\[
	q_2 : \textbf{SELECT MAX}(c_1) \textbf{ FROM } t_1 \textbf{ WHERE } c_2 < 200
	\]
	
	In the query structure, let \( n_1 \) and \( n_2 \) denote the \texttt{WHERE} clause nodes of 
	\( q_1 \) and \( q_2 \), respectively.
	The condition \( c_2 < 100 \) in \( n_1 \) is stricter than \( c_2 < 200 \) in \( n_2 \),
	so the rows selected by \( n_1 \) form a subset of those selected by \( n_2 \).
	The parent node of these filters is the aggregation operator \texttt{MAX},
	which is monotone increasing:
	when more rows are included, the maximum value of \( c_1 \) can only increase or remain the same.
	As a result, the difference at the set level (fewer or more tuples)
	propagates upward to a difference at the value level (smaller or larger aggregated value).
	
	Intuitively, \( n_1 \) and \( n_2 \) illustrate how a local change in the filter condition
	at the set level can influence the aggregated result value, demonstrating the propagation from Set to Value.
\end{example}


\begin{example}[Value $\rightarrow$ Set Propagation]
	Consider two semantically related queries over a table \( t_1(c_1) \):
	\[
	\begin{aligned}
		q_1 : &\; \textbf{SELECT} * \textbf{ FROM } 
		(\textbf{SELECT MAX}(c_1) \textbf{ AS } x \\ 
		&\textbf{FROM } t_1) \textbf{ AS } subq 
		\textbf{ WHERE } x > 100
	\end{aligned}
	\]
	
	\[
	\begin{aligned}
		q_2 : &\; \textbf{SELECT} * \textbf{ FROM } 
		(\textbf{SELECT MIN}(c_1) \textbf{ AS } x \\ 
		&\textbf{FROM } t_1) \textbf{ AS } subq 
		\textbf{ WHERE } x > 100
	\end{aligned}
	\]
	
	The two queries differ only in the inner aggregation.
	Let \( n_1 \) and \( n_2 \) denote the aggregation nodes 
	\texttt{MAX}(c\_1) and \texttt{MIN}(c\_1), respectively.
	Changing \texttt{MAX} to \texttt{MIN} decreases the derived value \( x \).
	Since the outer predicate \texttt{x > 100} is monotone increasing in \( x \),
	smaller \( x \) values make the condition harder to satisfy,
	resulting in fewer output tuples.
	Consequently, the result of \( q_2 \) becomes a subset of \( q_1 \),
	showing a typical Value $\rightarrow$ Set propagation.
\end{example}


These propagation behaviors connect the two approximation dimensions,
allowing a single mutation at any AST node (e.g., \texttt{MAX}$\rightarrow$\texttt{MIN})
to yield predictable, analyzable effects on both result structure and result values.
The unified propagation model forms the semantic foundation of our testing framework.

\section{Approach}

\subsection{Overview}

\subsection{Construction of the Original Query}

\subsection{Approximate Mutators}

\subsection{Approximation Propagation Analysis}

In this section, we further develop an executable algorithm to determine how local semantic changes propagate through the SQL AST.
The main purpose of this algorithm is to formalize the top-down reasoning process introduced in Definition~\ref{def:approximation_propagation} into a systematic, bottom-up propagation procedure that connects local node mutations with their global semantic consequences at the query level.

% Step 1: Initilization
Algorithm~\ref{alg:approximation_propagation} presents the overall propagation process.  
Initially (Line~1–3), the algorithm receives the mutated node information $node\_info$ and the complete AST of the query.  
It first identifies the mutated node $n_{mut}$ and constructs its ancestor chain from the mutation site to the query root, represented as $\text{list}=[n_{mut},\ldots,n_{root}]$.  
This structure enables the algorithm to traverse each parent operator sequentially and reason about how the mutation propagates upward.
% Step 2: Local relation at mutation site
The algorithm then initializes the local semantic level of the mutation (Line~4–6):
if the mutated node involves predicates or subqueries, it starts at the set level ($\alpha(n_{mut})=s$); otherwise, for expressions or aggregations, it starts at the value level ($\alpha(n_{mut})=v$).
A direction accumulator $sign$ is also initialized to $+1$ to indicate that propagation initially preserves directionality.
% Step 3: Bottom-up propagation
In the propagation stage (Line~8–17), the algorithm iteratively traverses each parent node of $n_{mut}$ in a bottom-up manner.  
For each parent operator $op_i$ (Line~9),  it consults Table~\ref{tab:op_type_rules} to retrieve the corresponding semantic mapping $(\alpha_{in} \rightarrow \alpha_{out})$ and its monotonic direction $\sigma(op_i)\in\{+1,-1\}$ (Line~10–12). 
The semantic level $\alpha$ is updated according to the operator’s input-output mapping (e.g., Set$\rightarrow$Value for aggregation or Value$\rightarrow$Set for predicate filters), while the cumulative direction $sign$ is updated multiplicatively ($sign \gets sign \cdot \sigma(op_i)$), preserving or reversing the relation based on operator polarity (Line~13–15).  
This recursive process effectively tracks the path of semantic transformation from the mutation site to the query output.
% Step 4: Materialize root-level relation
Finally, at the root node (Line~18–25), the algorithm derives the final query-level relation—returning either a set-level ($R(q_1,D)!\preceq^s_D!R(q_2,D)$) or value-level ($V_{q_1}(g,c)!\preceq^v_D!V_{q_2}(g,c)$) approximation, determined by the accumulated propagation direction.

\begin{algorithm}[t]
	\caption{Approximation Propagation across SQL AST}
	\label{alg:approximation_propagation}
	\footnotesize
	\begin{algorithmic}[1]
		\Require Mutated node info $node\_info$, SQL AST $AST$
		\Ensure Final query-level approximation between original and mutated queries
		
		\State \textbf{ // Step 1: Initialization}
		\State Identify mutated node $n_{mut}$.
		\State Build ancestor chain $\text{list}=[n_{mut},\ldots,n_{root}]$.
		
		\State \textbf{ // Step 2: Local relation at mutation site}
		\State Decide initial level $\alpha(n_{mut}) \in \{s,v\}$ by node type.
		\State Set local relation $\mathcal{R}(n_{mut}) \gets (\preceq^{\alpha(n_{mut})})$.
		\State Set direction accumulator $sign \gets +1$.
		
		\State \textbf{ // Step 3: Bottom-up propagation}
		\For{each parent node $n_i$ in \textit{list} (from child to root)}
		\State Determine operator type $op_i$ at $n_i$.
		\State Lookup rule in Table~\ref{tab:op_type_rules} for $op_i$.
		\State Obtain mapping $(\alpha_{in} \!\rightarrow\! \alpha_{out})$.
		\State Obtain monotonic direction $\sigma(op_i)\!\in\!\{+1,-1\}$.
		\State Update level: $\alpha(n_{i+1}) \gets \alpha_{out}$.
		\State Update direction: $sign \gets sign \cdot \sigma(op_i)$.
		\State Propagate symbolically: 
		\Statex \hspace{1.7em} $\mathcal{R}(n_{i+1}) \gets \mathcal{R}(n_i)$ with $sign$ applied.
		\EndFor
		
		\State \textbf{ // Step 4: Materialize root-level relation}
		\If{$\alpha(n_{root}) = s$}
		\State \textbf{return} $R(q_1,D) \preceq^{s}_{D} R(q_2,D)$ with sign $+$,
		\Statex \hspace{2.95em} or $R(q_1,D) \succeq^{s}_{D} R(q_2,D)$ with sign $-$.
		\Else
		\State \textbf{return} $V_{q_1}(g,c) \preceq^{v}_{D} V_{q_2}(g,c)$ with sign $+$,
		\Statex \hspace{2.95em} or $V_{q_1}(g,c) \succeq^{v}_{D} V_{q_2}(g,c)$ with sign $-$.
		\EndIf
	\end{algorithmic}
\end{algorithm}



\begin{table*}[t]
	\centering
	\scriptsize
	\caption{\textbf{Operator Type and Semantic Propagation Rules}}
	\vspace{-3mm}
	\label{tab:op_type_rules}
	\renewcommand{\arraystretch}{1.1}
	\begin{tabular}{p{2.5cm} p{2.7cm} p{2.7cm} p{6.5cm}}
		\toprule
		\textbf{Operator Type} & \textbf{Semantic Mapping} & 
		\textbf{Monotonic Direction $\sigma(op)$} & 
		\textbf{Example and Interpretation} \\
		\midrule
		
		\textbf{Aggregation} &
		Set $\rightarrow$ Value &
		\begin{tabular}[c]{@{}l@{}}$+1$: MAX, SUM, COUNT\\$-1$: MIN\end{tabular} &
		Input set expands $\Rightarrow$ aggregated value increases (MAX/SUM) or decreases (MIN). \\
		
		\midrule
		\textbf{Predicate / Filter} &
		Value $\rightarrow$ Set &
		\begin{tabular}[c]{@{}l@{}}$+1$: $x>c$, $x\ge c$\\$-1$: $x<c$, $x\le c$\end{tabular} &
		Increasing value makes predicate easier ($x>c$) or harder ($x<c$) to satisfy, thus expanding or shrinking result set. \\
		
		\midrule
		\textbf{Logical Operator} &
		Set $\rightarrow$ Set &
		\begin{tabular}[c]{@{}l@{}}$+1$: EXISTS\\$-1$: NOT, NOT EXISTS\end{tabular} &
		Negation flips inclusion direction (\texttt{EXISTS} $\leftrightarrow$ \texttt{NOT EXISTS}). \\
		
		\midrule
		\textbf{Arithmetic / Expression} &
		Value $\rightarrow$ Value &
		Depends on sign &
		\texttt{x + 1} ($+1$) preserves direction; \texttt{-x} ($-1$) reverses monotonicity. \\
		
		\midrule
		\textbf{Join / Projection / Subquery} &
		Set $\rightarrow$ Set &
		$+1$ &
		Structural operators preserve tuple inclusion without reversing semantic direction. \\
		
		\bottomrule
	\end{tabular}
\end{table*}


\subsection{Results Checking}


%\input{implement.tex}
%\section{Evaluation}
\label{sec:evaluation}

\subsection{Experimental Setup}

\subsection{Bug Detection}

\subsection{Bug Diversity}
% 哪些是由集合关系检测到的、哪些是由值检测到的,为什么现有的工具Pinolo检测不到。

\subsection{Comparative Study}
%\input{discussion.tex}
%\input{Limitations and Future work}
%\section{Related Work}

\textbf{Logical Bug Detection in DBMSs.}
A variety of approaches have been proposed to detect logical bugs in DBMSs~\cite{rigger2020detecting,rigger2020finding,rigger2020testing,song2023testing,tang2023detecting,jiang2024detecting,song2024detecting,hao2023pinolo,deng2025detecting}.  
\textsc{NoREC}~\cite{rigger2020detecting} transforms an optimizable query into a non-optimizable form and detects semantic inconsistencies by comparing their outputs.  
\textsc{TLP}~\cite{rigger2020finding} divides query predicates into multiple subqueries and verifies that the union of their results is equivalent to the original query.  
\textsc{PQS}~\cite{rigger2020testing} constructs queries expected to retrieve a specific pivot row, while \textsc{DQE}~\cite{song2023testing} detects logical bugs by comparing whether different SQL queries with the same predicate access the same rows in the database.
\textsc{TQS}~\cite{tang2023detecting} decomposes wide tables into smaller ones and uses the base table as ground truth for correctness validation.  
\textsc{EET}~\cite{jiang2024detecting} applies expression-preserving transformations, and \textsc{Radar}~\cite{song2024detecting} compares query results between databases with and without metadata to identify semantic flaws.  
\textsc{EDC}~\cite{deng2025detecting} detects logical bugs by substituting expressions with precomputed equivalent data and checking for inconsistent results, while \textsc{CODDTest}~\cite{zhang2025constant} leverages constant folding and propagation to generate equivalent queries.  

Most of these methods rely on constructing \textbf{equivalent SQL pairs}, detecting bugs when such equivalence fails to hold.  
Recently, \textsc{Pinolo}~\cite{hao2023pinolo} generalizes this paradigm through set-level approximation, where predicates are relaxed or restricted to generate over- and under-approximate queries, and correctness is judged by inclusion or containment relations between result sets.  
%\textit{SRS}~\cite{} further explores semantic relations by transforming join queries—modifying join types, orders, and conditions—while maintaining consistent set-level semantics.  
Building on this foundation, our work extends the set semantics to the value semantics level, enabling the detection of subtler logical errors that cannot be captured by set inclusion alone.  
By combining set-level consistency with value-directional reasoning, our approach establishes a multi-dimensional criterion for identifying query approximations and uncovering deeply hidden semantic faults.


\textbf{DBMS Test-Case Generation.}
A variety of approaches have been proposed for generating diverse test cases for DBMSs, with the aim of improving coverage and revealing potential bugs. 
These techniques~\cite{sqlsmith,zhong2020squirrel,fu2022griffin,jiang2023dynsql} typically focus on generating valid and varied SQL queries, but may not specifically target logical bugs. 
\textsc{SQLsmith}~\cite{sqlsmith} is a grammar-based DBMS fuzzer that embeds SQL grammar rules to generate complex SQL queries. 
It uses a random walk approach to explore the SQL syntax and generate a wide range of queries.
\textsc{SQUIRREL}~\cite{zhong2020squirrel} is a mutation-based DBMS fuzzer which introduces an intermediate representation for SQL queries and models dependencies between SQL statements, enabling the generation of queries that contain multiple SQL operations. 
\textsc{Griffin}~\cite{fu2022griffin} uses a grammar-free mutation approach, where it mutates SQL queries based on DBMS state information encapsulated in a metadata graph. 
\textsc{QTRAN}~\cite{lin2025qtran} is a LLM-based approach that can automatically translate test cases from other DBMSs.
\textsc{DynSQL}~\cite{jiang2023dynsql} takes a dynamic approach by interacting with the DBMS to capture the latest state information, allowing for the incremental generation of valid and complex queries.
These techniques aim to prevent semantic errors and improve the diversity of generated queries. 
In addition to general query generation, some approaches have been specifically designed to aid in bug detection. 
\textsc{SQLRight}~\cite{liang2022detecting} leverages code coverage feedback to enhance test-case generation. 
This feedback provides insights into which parts of the DBMS code are exercised, increasing the chances of uncovering logical bugs in infrequently executed paths. 
\textsc{QPG}~\cite{ba2023testing} takes a different approach by recording the query plans covered during DBMS testing and prioritizing the mutation of queries that trigger new query plans. 
This targeted mutation is more likely to expose logical bugs that might otherwise remain undetected.

These general query generation approaches complement \toolname. 
While approaches like \textsc{SQLsmith}, \textsc{SQUIRREL}, and \textsc{Griffin} focus on generating diverse queries, \toolname\ can help identify logical bugs hidden in complex or rarely executed query logic. 
Conversely, the test cases generated by these methods can provide \toolname\ with high-quality and varied queries, expanding its ability to uncover bugs. 
Together, these approaches offer a comprehensive strategy for DBMS testing, improving both query coverage and the detection of logical bugs.








%\input{conclusion.tex}


%\newpage

\bibliographystyle{ACM-Reference-Format}
\balance
\bibliography{sample-base}
%\printbibliography

\end{document}
\endinput